\documentclass{article}
\usepackage{listings} % For code listing
\usepackage{xcolor} % For syntax highlighting
\usepackage{geometry} % For adjusting margins
\usepackage{titling} % For adjusting title
\usepackage{sectsty} % For section font styling
\usepackage{enumitem} % For adjusting list item spacing

% Define custom colors
\definecolor{codebg}{rgb}{0.95,0.95,0.95}
\definecolor{darkblue}{rgb}{0.0,0.0,0.4}

% Set code listing style
\lstset{
    backgroundcolor=\color{codebg},
    basicstyle=\small\ttfamily,
    keywordstyle=\color{blue},
    stringstyle=\color{red},
    commentstyle=\color{green!60!black},
    breaklines=true,
    showstringspaces=false,
    captionpos=b,
    numbers=left,
    numbersep=5pt,
    numberstyle=\tiny
    \color{gray},
}

% Adjust margins
\geometry{margin=1in}

% Adjust title formatting
\pretitle{\begin{center}\LARGE\bfseries}
\posttitle{\par\end{center}\vskip 0.5em}

% Section font styling
\sectionfont{\color{darkblue}}

% Adjust list item spacing
\setlist{nosep}

\title{Differences Between Normal and Static Functions in C++}
\author{S.A. Abdulla}
\date{21/01/2024}

\begin{document}
\maketitle

\newpage
\section*{Introduction}
In C++, you can't directly catch a variable like \texttt{catch(k)}. 
When catching exceptions, you need to specify the type of the exception being caught;
you can't use only the variable name to catch.
\newline
For example, if you want to catch an exception of type \texttt{int},
you would typically catch it by reference:

\begin{lstlisting}[language=C++, caption=Example of catching an int exception]
int k;
try {
    // code that may throw an int
} catch (int &k) {
    // handle the exception
}
\end{lstlisting}

In C++, a \texttt{throw} statement can only throw one exception object at a time. Each \texttt{throw} statement typically throws one instance of a particular exception type.

\section*{Custom Exception Handling}
\begin{lstlisting}[language=C++, caption=Example of custom exception handling]
#include <iostream>
using namespace std;

class CustomException {
public:
    int errorCode;
    string errorMessage;

    CustomException(int code, const string& message) : errorCode(code), errorMessage(message) {}
};

void foo() {
    throw CustomException(404, "Not Found");
}

int main() {
    try {
        foo();
    }
    catch(const CustomException& e) {
        cout << "Error Code: " << e.errorCode << endl;
        cout << "Error Message: " << e.errorMessage << endl;
    }
    return 0;
}
\end{lstlisting}
\newpage

\section*{Handling Multiple Exceptions}
If you want to handle multiple exceptions, it has to be done separately. If we use \texttt{throw} one after another without catching, only one of the \texttt{catch} blocks will execute.

\begin{lstlisting}[language=C++, caption=Example of handling multiple exceptions]
    #include <iostream>

    using namespace std;
    
    // Define the Exception class
    class MyException {
    public:
        int value;
        bool isPtrException;
    
        MyException(int val, bool isPtr) : value(val), isPtrException(isPtr) {}
    };
    
    // Define the fun function
    void fun(int* ptr, int val) {
        if (ptr == nullptr) {
            throw MyException(val, true);
        } else {
            throw MyException(val, false);
        }
    }
    
    int main() {
        int y = 0;
        try {
            fun(nullptr, y); // Throws MyException with isPtrException=true
        } catch (MyException &e) {
            if (e.isPtrException) {
                cout << "Caught exception from ptr value: " << e.value << endl;
            } else {
                cout << "Caught exception from val value: " << e.value << endl;
            }
        }
    
        try {
            int x = 0;
            fun(&x, y); // Throws MyException with isPtrException=false
        } catch (MyException &k) {
            if (k.isPtrException) {
                cout << "Caught exception from ptr value: " << k.value << endl;
            } else {
                cout << "Caught exception from val value: " << k.value << endl;
            }
        }
    
        return 0;
    }
    
\end{lstlisting}

\end{document}
